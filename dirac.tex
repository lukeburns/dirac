\documentclass{article}
\usepackage[margin=1.25in]{geometry}
\usepackage{amsmath}
\usepackage{amsfonts}

\title{An Extension of the Dirac Equation}
\author{Luke Burns}

\begin{document}
  \maketitle

  \section{Introduction}

  The free Dirac equation in STA is

  \begin{equation}
    \nabla \psi I \sigma_3 = \psi p_0, \label{eq:dirac}
  \end{equation}

  where

  \begin{equation}
    p_0 = m \gamma_0. \label{eq:mass}
  \end{equation} 

  A less restrictive constraint is just that

  \begin{equation}
    p_0^2 = m^2
  \end{equation}

  be constant. With this lightened restriction, Equation \ref{eq:dirac} admits a new class of null solutions, where $p_0^2 = 0$ but $\psi p_0 \not = 0$.

  \section{General Results}

\subsection{Constraints on $p_0$}

  The extension of the full Dirac equation, including the electromagnetic gauge field

  \begin{equation}
    \nabla \psi I \sigma_3 - e A \psi = \psi p_0 \label{eq:full}
  \end{equation}

  is only invariant under the usual replacements $\psi \mapsto \psi e^{\alpha I \sigma_3}$ and $A \mapsto A - \nabla \alpha$ if $p_0 \cdot I\sigma_3 = 0$:

  \begin{align}
    \nabla (\psi e^{\alpha I \sigma_3}) I \sigma_3 - e (A - \nabla \alpha) \psi e^{\alpha I \sigma_3} &= (\nabla \psi I \sigma_3  - e A \psi) e^{\alpha I \sigma_3} \\
     &= \psi p_0 e^{\alpha I \sigma_3} \\
     &= \psi e^{\alpha I \sigma_3} p_0.
  \end{align}

  This is an essential feature of Dirac theory, so we'll restrict $p_0$ preserve it. We require that

  \begin{equation}
    p_0 = E \gamma_0 \pm |\vec p| \gamma_3 = (E + \vec p) \gamma_0.
  \end{equation} 

  If $p_0^2 = m^2 \not= 0$, then 

  \begin{equation}
    p_0 = R m \gamma_0 \widetilde R
  \end{equation} 

   for $R = e^{\gamma_3 \gamma_0 \alpha/2}$ with $\alpha$ given by $\tanh(\alpha) = |\vec p|/E$. Then $p_0 \mapsto \widetilde R p_0 R$ reduces to Equation \ref{eq:mass}.

  On the other hand, if $p_0^2 = 0$, then

  \begin{equation}
    p_0 = \omega (1 + \hat p) \gamma_0, \label{eq:massless}
  \end{equation}

  for $\omega = E = |\vec p|$, which gives an equation of the form

  \begin{equation}
    \nabla \psi I \sigma_3 = \omega \psi (1 \pm \sigma_3) \gamma_0.\label{eq:extension}
  \end{equation}

  Solutions to Equation \ref{eq:dirac} satisfying Equation \ref{eq:massless} are distinct from those satisfying Equation \ref{eq:mass}. In particular, notice that $\psi$ is an ordinary Dirac spinor --- not a Weyl spinor like the massless solutions to the Dirac equation. These solutions are distinct from the massless solutions to the Dirac equation and yield different observables. Most notably, as we'll show next, the usual probability current is not conserved.

  \subsection{The Probability Current}

  In general, if $v_0$ is a constant vector, then

  \begin{equation}
    \nabla \cdot (\psi v_0 \widetilde \psi) = \langle v_0 \wedge p_0 (I \sigma_3 \psi \widetilde \psi) \rangle + \langle v_0 \cdot I \sigma_3 (\widetilde \psi e A \psi) \rangle,
  \end{equation}

  and

  \begin{equation}
    \nabla \cdot (\psi v_0 \widetilde \psi) = 0 \iff v_0 \wedge p_0 = 0. 
  \end{equation}

  That is, energy-momentum density $\rho p = \psi p_0 \widetilde \psi$ is the only vector-valued bilinear covariant conserved generally (up to a constant multiple).

  In particular, notice that the conservation of the usual probability current $J = \psi \gamma_0 \widetilde \psi = \rho v$ does \emph{not} vanish for $p_0 = \omega (1 + \hat p) \gamma_0$, since $\gamma_0 \wedge p_0 = - \vec p = \mp \omega \sigma_3 \not= 0$ and

  \begin{align}
    \nabla \cdot J &= \langle \gamma_0 \wedge p_0 (I \sigma_3 \psi \widetilde \psi) \rangle \\
                   &= \mp \omega \rho \langle I e^{I \beta} \rangle \\
                   &= \pm \omega \rho \sin \beta.
  \end{align}

  A consequence is that the interpretation of $J$ as a probability current and the normalization procedure

  \begin{equation}
    \int d^3x \gamma_0 \cdot J = 1\label{eq:mnormal}
  \end{equation}

  do not carry over to Equation \ref{eq:massless}.\footnote{Well, the normalization procedure carries over perfectly well if $C = \omega$, as defined in Equation \ref{eq:normalization}.} However, we can extend this straightforwardly by rewriting things in terms of $p$. Equation \ref{eq:mnormal} is equivalent to

  \begin{equation}
    \int d^3x \rho \gamma_0 \cdot p = m,
  \end{equation}

  which simply ensures that integrating energy density over all of space, say, in the $\gamma_0$ frame, is just the rest energy of the particle.

  A reasonable generalization of this is

  \begin{equation}
    \int d^3x \rho \gamma_0 \cdot p = C. \label{eq:normalization}
  \end{equation}

  for a constant $C$. Any choice of $C$ determines a probability current $j = \psi p_0 \widetilde \psi / C$ with a normalized probability density $j_0 = \gamma_0 \cdot j$. For null $p_0$, however, there is no privileged frame that gives a natural choice of $C$. Selecting $C = \gamma_0 \cdot p_0$ for may be a convenient choice, because it coincides with $\omega$ in the massless theory, reducing the number of arbitrary parameters to be specified, and aligns with the usual choice $m$ in the massive theory.

\end{document}
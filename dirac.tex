\documentclass[twocolumn]{article}
\usepackage[margin=.7in]{geometry}
\usepackage{amsmath}
\usepackage{sectsty}
\usepackage{amsfonts}
\usepackage{multicol}
\usepackage{hyperref}

\sectionfont{\fontsize{12}{15}\selectfont}
\renewcommand{\abstractname}{\vspace{-\baselineskip}}

\title{An Extension of the Dirac Equation}
\author{Luke Burns}
\date{\small \today\vspace{-5ex}}

\begin{document}
\twocolumn[
\begin{@twocolumnfalse}
  \maketitle

  \abstract{A minimal extension of the Dirac equation is shown to describe a pair of massless, electrically charged fermions. Solutions exhibit rotation in the particle's spin plane, analogous to the zitterbewegung of massive Dirac theory and identical to electric and magnetic fields of circularly polarized electromagnetic waves.}

  \paragraph{Work In Progress} This paper is a work in progress and is being openly developed on Github at \url{https://github.com/lukeburns/dirac}. Contributions are warmly welcomed, whether by means of opening an issue or pull request. \\
\end{@twocolumnfalse} ]

  \section{Introduction}

  The free Dirac equation as expressed in the Space Time Algebra (STA) is\cite{gap}

  \begin{equation}
    \nabla \psi I \sigma_3 = \psi p_0, \label{eq:dirac}
  \end{equation}

  where

  \begin{equation}
    p_0 = m \gamma_0. \label{eq:mass}
  \end{equation} 

  This equation in part describes the dynamics of a spinor $\psi = \rho^{1/2} e^{I \beta/2}R$ that rotates, boosts, and dilates the momentum of a particle in its rest frame $p_0 = m \gamma_0$ onto a probability current $mJ = \psi p_0 \widetilde \psi$ in some other frame. However, this description fails for massless particles, since they have no rest frame.

  A minimal extension to this equation that accomodates a similar description for massless particles is to simply require that $p_0$ be constant, so that Equation \ref{eq:dirac} describes the dynamics of a spinor that rotates, boosts, and dilates the momentum $p_0$ of a particle in some arbitrary ``initial'' frame onto a probability current $\psi p_0 \widetilde \psi$ in some other frame (it will be shown that this works in general in Section \ref{probability}).

  The extension including the electromagnetic gauge field can be written

  \begin{equation}
    \nabla \psi I \sigma_3 - e A \psi = \psi p_0, \label{eq:full}
  \end{equation}

  where $p_0$ is constant.

  In Section \ref{constraints}, we'll work out the physical constraints on $p_0$ and show that Equation \ref{eq:full} admits the usual solutions to the Dirac equation and a new class of null solutions, where $p_0^2 = 0$ but $\psi p_0 \not = 0$. Then we'll work out the symmetries of Equation \ref{eq:full} in Section \ref{symmetries}. We will show that a probablistic interpretation extends to Equation \ref{eq:full} in Section \ref{probability} and present plane wave solutions in Section \ref{waves}. Lastly, an apparent similarity to circularly polarized electromagnetic waves is formalized in Section \ref{electromagnetism}. We will set $\hbar = c = 1$ throughout.

  % The Matrix formulation is tucked away in Appendix \ref{matrix}. I encourage the reader to check out STA, which provides the dynamical spinor description used above to motivate the extension. See \url{https://github.com/ga/Resources} for a collection of resources.

  \section{Constraints}\label{constraints}

  There are two constraints that we must impose on $p_0$. Firstly, Equation \ref{eq:full} must be invariant under gauge transformations if it is to be compatible with the gauge field $A$, and secondly, solutions must satisfy the Klein-Gordon equation.

  Equation \ref{eq:full} is only gauge invariant under the replacements $\psi \mapsto \psi e^{\alpha I \sigma_3}$ and $A \mapsto A - \nabla \alpha$ if 

  \begin{equation}
    p_0 I\sigma_3 = I \sigma_3 p_0. \label{momentum-perp-spin}
  \end{equation} 

  That is, if $p_0$ is perpendicular to the plane $I \sigma_3$, so $p_0$ is of the form

  \begin{equation}
    p_0 \equiv p_0^\pm = E_0 \gamma_0 \pm |\vec p_0| \gamma_3,\label{eq:p_0}
  \end{equation} 

  where $E_0$ and $\vec p_0 = \pm |\vec p_0| \gamma_3 \gamma_0$ are constant.

  Additionally, if solutions are to satisfy the Klein-Gordon equation, we must have 

  \begin{equation}
    p_0^2 \geq 0.
  \end{equation}

  If $p_0^2 > 0$, then 

  \begin{equation}
    p_0 = R m \gamma_0 \widetilde R\label{eq:R}
  \end{equation} 

  for $R = e^{\pm \gamma_3 \gamma_0 \alpha/2}$ with $\alpha$ given by $\tanh(\alpha) = |\vec p_0|/E_0$. This implies that if $\psi$ is a spinor valued solution to Equation \ref{eq:full}, then $\psi R$ is a solution to the Dirac equation. On the other hand, if $\psi$ is instead a solution to the Dirac equation, then $\psi R$ and $\psi \widetilde R$ are solutions to Equation \ref{eq:full}. That is, spinor valued solutions to Equation \ref{eq:full} are in two-to-one correspondence to solutions of the Dirac equation. 

  We will see in Section \ref{symmetries} that a general solution to Equation \ref{eq:full} is a full multivector. However, the full multivector solutions are equivalent to a pair of decoupled spinor valued solutions to Equations \ref{eq:even} and \ref{eq:odd}, corresponding precisely to Equation \ref{eq:full} for each sign of $\vec p_0 = \pm |\vec p_0| \gamma_3 \gamma_0$.

  This two-to-one correspondence is reflective of the fact that one can always find two frames in which a massive particle is propagating with the same momentum but in opposite directions. This is the reason helicity (the projection of the momentum onto spin) is not a Lorentz invariant quantity for massive particles but is for massless particles. On the other hand, given a massless particle propagating in one direction, there does not exist a frame in which that particle travels in the opposite direction. Helicity \emph{is} Lorentz invariant for massless particles for this reason.

  Unlike the massive case, there is no correspondence between massless solutions to Equation \ref{eq:full} and the Dirac equation.  If $p_0^2 = 0$, then $p_0$ can be written

  \begin{equation}
    p_0 = \omega_0 (1 \pm \sigma_3) \gamma_0, \label{eq:massless}
  \end{equation}

  where $\omega_0 \equiv E_0 = |\vec p_0|$. The decomposition

  \begin{equation}
    \psi = \psi \frac{1 + \sigma_3}{2} + \psi \frac{1 - \sigma_3}{2} = \psi_+ + \psi_-,
  \end{equation}

  implies that if $\psi$ is a solution to Equation \ref{eq:full}, then the Weyl spinor $\psi_\pm$ is a solution to the massless Dirac equation. We can see this by multiplying Equation \ref{eq:full} by $(1 \pm \sigma_3)/2$ on the right, which gives

  \begin{equation}
    \nabla \psi_\pm I \sigma_3 - e A \psi_\pm = \frac{1}{2} \omega_0 \psi (1 \pm \sigma_3)(1 \mp \sigma_3) \gamma_0 = 0,
  \end{equation}

  using the facts that $I \sigma_3 (1 \pm \sigma_3) = (1 \pm \sigma_3) I \sigma_3$, $\gamma_0 (1 \pm \sigma_3) = (1 \mp \sigma_3)\gamma_0$, and $(1 \pm \sigma_3) (1 \mp \sigma_3) = 0$.

  However, $\psi_\mp$ is simply another solution to Equation \ref{eq:full} and \emph{not} a solution to the Dirac equation. This can be seen by multiplying Equation \ref{eq:full} on the right by by $(1 \mp \sigma_3)/2$, which gives

  \begin{equation}
    \nabla \psi_\mp I \sigma_3 - e A \psi_\mp = \frac{1}{2} \omega_0 \psi(1 \pm \sigma_3)(1 \pm \sigma_3) \gamma_0 = \psi p_0,
  \end{equation}

  since $\frac{1}{2}(1 \pm \sigma_3)(1 \pm \sigma_3) = (1 \pm \sigma_3)$.

  Since the projection operator $(1 \pm \sigma_3)/2$ is not invertible, there is no way to recover solutions to Equation \ref{eq:full} from solutions to the massless Dirac equation. Hence, Equation \ref{eq:full} contains null solutions that are distinct from solutions to the massless Dirac equation and, furthermore, we have shown that these are the \emph{only} new solutions admitted by the extension. In this sense, the extension is minimal.

  \section{Symmetries} \label{symmetries}

  In this section, we'll show that solutions to Equation \ref{eq:full} are, in general, multivectors but that a multivector solution is just a pair of independent spinor valued solutions to Equation \ref{eq:full}. Then we'll construct charge, parity, and time reversal conjugations.

  Unlike the Dirac equation, Equation \ref{eq:full} is not invariant under

  \begin{equation}
    \psi \mapsto \psi \gamma_0.\label{eq:g0conjugation}
  \end{equation}

  If $\psi$ is a solution to

  \begin{equation}
    \nabla \psi I \sigma_3 - e A \psi = \psi p_0,\label{eq:plus}
  \end{equation}

  then $\psi' = \psi \gamma_0$ is a solution to

  \begin{equation}
    \nabla \psi' I \sigma_3 - e A \psi' = \psi' \overline p_0, \label{eq:minus}
  \end{equation}

  where $\overline M \equiv \gamma_0 M\gamma_0$ is minus the reflection of any multivector $M$ across the $\gamma_0$ axis. That is to say, Equation \ref{eq:full} distinguishes between even and odd fields. 

  In general, $\psi$ is a full multivector that can be decomposed into

  \begin{equation}
    \psi = \langle \psi \rangle_+ + \langle \psi \rangle_-,
  \end{equation}

  where $\langle \psi \rangle_+$ and $\langle \psi \rangle_-$ are even and odd multivectors respectively that are independent solutions to

  \begin{equation}
    \nabla \langle \psi \rangle_\pm I \sigma_3 - e A \langle \psi \rangle_\pm = \langle \psi \rangle_\pm p_0,\label{eq:decoupled}
  \end{equation}

  because $\nabla$, $A$, and $p_0$ are all odd valued. If, say, the gauge field $A$ were even valued, then $\langle \psi \rangle_+$ and $\langle \psi \rangle_-$ would be coupled.\footnote{This raises the interesting possibility of having a spinor valued (i.e. fermionic) gauge field. The idea of a spinor valued gauge field came to my attention due to V\'aclav Zatloukal in private communications, independently of the content of this paper.}

  Equations \ref{eq:decoupled} can be re-expressed in terms of two equations involving only spinors. Doing this will allow for an straightforward comparison to Dirac theory. If $\psi_+ = \langle \psi \rangle_+$ is a solution to

  \begin{equation}
    \nabla \psi_+ I \sigma_3 - e A \psi_+ = \omega_0 \psi_+ (1 + \sigma_3) \gamma_0,\label{eq:even}
  \end{equation}

  then the even multivector 

  \begin{equation}
    \psi_- \equiv \langle \psi \rangle_- \gamma_0\label{eq:0decomposition}
  \end{equation} 

  is a solution to

  \begin{equation}
    \nabla \psi_- I \sigma_3 - e A \psi_- = \omega_0 \psi_- (1 - \sigma_3) \gamma_0,\label{eq:odd}
  \end{equation}

  and these are precisely Equation \ref{eq:full} for each sign of $p_0$.

  At first glance, it appears that Equations \ref{eq:even} and \ref{eq:odd} describe particles of opposite helicity. However, this is not exactly the case, because Equations \ref{eq:even} and \ref{eq:odd} each admit solutions with both positive and negative helicity. For instance, if $\psi$ is a solution to Equation \ref{eq:even}, then $I \psi$ is a solution to the same equation with opposite charge and helicity. The characteristic quantity that distinguishes between Equations \ref{eq:even} and \ref{eq:odd} is the ``projection'' of charge onto helicity, which one can see by inspecting Equations \ref{eq:even} and \ref{eq:odd}.

  This is an example of a conjugation: a mapping between solutions to an equation. The most important conjugations are charge (C), parity (P), and time reversal (T) conjugations. Combined CPT symmetry is a fundamental symmetry, the violation of which would indicate a violation of Lorentz invariance.

  For general multivector solutions, these conjugations are given by

  \begin{align}
    \hat C \psi = \psi \gamma_1 &\iff eA \mapsto - eA \label{eq:charge}\\
    \hat P \psi = \overline \psi(\overline x) &\iff \nabla \mapsto \overline \nabla, eA(x) \mapsto e\overline A(\overline x), \notag \\ & \hspace{3em}\text{and } p_0 \mapsto \overline p_0 \label{eq:parity}\\
    \hat T \psi = -I \overline \psi(-\overline x) \gamma_1 &\iff \nabla \mapsto \overline\nabla, eA(x) \mapsto -e\overline A(-\overline x), \notag \\ & \hspace{3em} \text{and } p_0 \mapsto - \overline p_0.\label{eq:time}
  \end{align}

  Parity and the combined CPT conjugation

  \begin{equation}
    \hat C \hat P \hat T \psi(x) = -I \psi(-x) \iff A(x) \mapsto A(-x) \text{ and } p_0 \mapsto -p_0 \label{eq:cpt}
  \end{equation}

  are grade preserving conjugations and so apply directly to $\psi_+$ and $\psi_-$ in Equation \ref{eq:even} and \ref{eq:odd}. However, charge and time reversal conjugations swap even and odd parts of $\psi$ and so transform $\psi_+ = \langle \psi \rangle_+$ and $\psi_- = \langle \psi \rangle_- \gamma_0$ differently.

  Under charge conjugation, $\langle \psi \rangle_\pm$ transform as

  \begin{align}
    \psi \mapsto \psi \gamma_1 \implies 
    &\langle \psi \rangle_+ \mapsto \langle \psi \rangle_+ \gamma_1 = \langle \psi \gamma_1 \rangle_- \\
    \text{ and } 
    &\langle \psi \rangle_- \mapsto \langle \psi \rangle_- \gamma_1 = \langle \psi \gamma_1 \rangle_+.
  \end{align}

  So $\psi_+$ and $\psi_-$ transform as

  \begin{equation}
    \psi_+ \mapsto \psi_-' = \psi_+ \sigma_1
    \text{ and } 
    \psi_- \mapsto \psi_+' = - \psi_- \sigma_1,
  \end{equation}

  where $\psi_-' \equiv \langle \psi' \rangle_- \gamma_0$ and $\psi_+' \equiv \langle \psi' \rangle_+$, which induces a sign change in both charge and helicity.

  Therefore, C, P, and T conjugations for $\psi_+$ and $\psi_-$ are given by

  \begin{align}
    \hat C \psi_\pm = \pm \psi_\pm \sigma_1 &\iff eA \mapsto - eA \text{ and } p_0 \mapsto \overline p_0. \label{eq:pmcharge}\\
    \hat P \psi_\pm = \overline \psi_\pm(\overline x) &\iff \nabla \mapsto \overline \nabla, eA(x) \mapsto e\overline A(\overline x), \notag \\ & \hspace{3em} \text{and } p_0 \mapsto \overline p_0. \label{eq:pmparity} \\
    \hat T \psi_\pm = \mp I \overline \psi_\pm(-\overline x) \sigma_1 &\iff \nabla \mapsto \overline\nabla, eA(x) \mapsto -e\overline A(-\overline x), \notag \\ & \hspace{3em} \text{and } p_0 \mapsto - p_0.\label{eq:pmtime}
  \end{align}

  Note that if $A = 0$, then charge conjugation provides a map between solutions to Equations \ref{eq:even} and \ref{eq:odd}. Otherwise, they are distinct, and describe two distinct particles: one with correlated charge and helicity, one with anti-correlated charge and helicity.

  \section{Probability Current}\label{probability}

  Essential to Dirac theory is its probabilistic intepretation, which depends on a conserved probability current $J$ satisfying the continuity equation

  \begin{equation}
    \nabla \cdot J = 0.
  \end{equation}

  The usual probability current $\psi \gamma_0 \widetilde \psi$ of Dirac theory is not conserved for spinor valued solutions $\psi$ to Equation \ref{eq:full}. To see this, consider the following, for a constant vector $v_0$.

  \begin{equation}
    \nabla \cdot (\psi v_0 \widetilde \psi) = \langle v_0 \wedge p_0 (I \sigma_3 \psi \widetilde \psi) \rangle + \langle v_0 \cdot I \sigma_3 (\widetilde \psi e A \psi) \rangle,\label{eq:current-expansion}
  \end{equation}

  which gives a condition for conservation

  \begin{equation}
    \nabla \cdot (\psi v_0 \widetilde \psi) = 0 \iff v_0 \wedge p_0 = 0.
  \end{equation}

  The second term in Equation \ref{eq:current-expansion} vanishes, because $v_0 \wedge p_0 = 0$ implies $v_0 \cdot I \sigma_3 = 0$, due to Equation \ref{momentum-perp-spin}.

  This means that $\psi p_0 \widetilde \psi$ is the only vector valued bilinear covariant conserved in general (up to a constant multiple). Furthermore, the fact that $\nabla \cdot (\psi p_0 \widetilde \psi) = 0$ implies the existence of streamlines with tangents given by $p = R p_0 \widetilde R$, which are timelike if $p_0^2 > 0$ and lightlike if $p_0^2 = 0$.\cite{hestenes} The usual probability current $\psi \gamma_0 \widetilde \psi$ is not conserved because $\gamma_0 \wedge p_0 \not= 0$.

  The normalization procedure

  \begin{equation}
    \int d^3x \gamma_0 \cdot J = 1\label{eq:mnormal}
  \end{equation}

  can be extended straightforwardly. In Dirac theory, Equation \ref{eq:mnormal} is equivalent to

  \begin{equation}
    \int d^3x \gamma_0 \cdot (\psi p_0 \widetilde \psi) = m,
  \end{equation}

  which simply ensures that integrating energy density (in the $\gamma_0$ frame) over all of space is just the rest energy of the particle.

  Since massless particles do not have rest energy, a reasonable generalization of this is

  \begin{equation}
    \int d^3x \gamma_0 \cdot (\psi p_0 \widetilde \psi) = c. \label{eq:normalization}
  \end{equation}

  for a constant $c$. Any choice of $c \not= 0$ determines a probability current $J = \psi p_0 \widetilde \psi / c$ with a normalized probability density $J_0 = \gamma_0 \cdot J$. Selecting $c = \gamma_0 \cdot p_0$ may be a convenient choice, because it coincides with $\omega_0$ when $p_0^2 = 0$ and aligns with the usual choice $m$ when $p_0^2 > 0$. Alternatively, there's nothing preventing us from choosing $c = 1$ and simply referring to $\rho p = \psi p_0 \widetilde \psi$ as the probability current.

  % \section{Observables}

  % What is the appropriate definition of spin? The spin vector of hestenes and spin bivector of doran/lasenby are incompatible, due to duality factor. 

  % Why does Hestenes use a lightlike spin bivector? Is there a relation between this bivector and F below?

  % If I were to reverse engineer the process, following the strategy of Vaz, seeking a spinorial description of circularly polarized electromagnetic fields, would I end up with the same equation?
  %   - This would contribute to the Vaz paper, which depended on invertibility.

  % That is, are all circularly polarized electromagnetic waves described via this equation?

  % Note other forms of the Equation \ref{eq:dirac}:

  % \begin{align}
  %   \nabla \psi I \sigma_3 &= \rho^{1/2} R' p_0 = p \psi \\
  %   \nabla \psi &= p s^{-1}\psi = I \sigma_3' p \psi = I p \psi \\
  %   -I \nabla \psi &= \psi p_0 \\
  %   &= \rho^{1/2}(x) e^{I \beta'(x)} R_0
  % \end{align}

  % which have different symmetries. In particular, the second form requires that the only kinematical part is due to $\rho^{1/2} e^{I \beta/2}$ (otherwise, there would be a bivector term in the derivative). This has fewer degrees of freedom than Equation \ref{eq:full}, doesn't it? $R_0$ is constant and can pick out any frequency for $p$. 

  % Gauging: 

  % \begin{align}
  %   -I \nabla \psi &= p \psi.\\
  %   \psi &\mapsto \psi e^{I \phi(x)} \\
  %   &\implies\\
  %   -I \nabla \psi' &= -I \nabla \psi e^{I \phi} - \nabla \phi \psi e^{I \phi}\\
  %   &\implies
  %   D \psi = -I \nabla \psi - eA \psi = p \psi,
  % \end{align}

  % where $A \mapsto A - \nabla \phi$.

  % If $\phi = \phi_e + I \phi_b$, then $A = A_e + I A_b$ gives

  % \begin{equation}
  %   F = \langle \nabla A \rangle_2 = F_e + F_b = \nabla \wedge A_e + \nabla \cdot I A_b
  % \end{equation}

  % such that

  % \begin{equation}
  %   \nabla F = J \equiv J_e + I J_b.
  % \end{equation}

  \section{Plane Waves} \label{waves}

  Plane wave solutions for Equation \ref{eq:even} can be written

  \begin{equation}
    \nabla \psi_+ I \sigma_3 = \pm p \psi_+,\label{eq:plane}
  \end{equation}
  where $p = \omega_0 R (\gamma_0 + \gamma_3) \widetilde R$ is constant and $\omega = p \cdot \gamma_0 > 0$.

  For any solution $\psi$, $\rho$ is constant, $\beta = 0$ or $\beta = \pi$, and $R = R_0 e^{\mp I \sigma_3 (p \cdot x + c)}$, where $R_0$ is constant and $c$ is a monogenic phase shift satisfying $\nabla c(x) = 0$. For every $\rho$ and $R_0$, which completely determine the constant probability current $\rho p = \psi p_0 \widetilde \psi = \rho R_0 p_0 \widetilde R_0$, we have two solutions (taking $c = 0$)

  \begin{equation}
    \psi^{(+)}_+(x) = \rho^{1/2} R_0 e^{- I \sigma_3 p \cdot x} \text { and } \psi^{(-)}_+(x) = \rho^{1/2} I R_0 e^{I \sigma_3 p \cdot x},
  \end{equation} 

  which are CPT conjugates of one another. $\psi^{(+)}_+$ and $\psi^{(-)}_+$ describe particles propagating in the $\vec p_0 = p \wedge \gamma_0$ direction (in the $\gamma_0$ frame) with opposite spin, given by 

  \begin{equation}
    S = \psi I \sigma_3 \widetilde \psi = (\pm)\rho R I \sigma_3 \widetilde R.
  \end{equation}

  There are two corresponding solutions for Equation \ref{eq:odd} of the form

  \begin{equation}
    \psi^{(+)}_- = \psi^{(+)}_+ \sigma_1 \text{ and }
    \psi^{(-)}_- = \psi^{(-)}_+ \sigma_1,
  \end{equation}

  which propagate in $-\vec p_0$ direction.

  These solutions are similar to the plane wave solutions to the Dirac equation in that there are four of them, each of which is described by a Dirac (four component) spinor. On the other hand, they are similar to Weyl plane waves, in that their momentum is lightlike and their helicity is Lorentz invariant.

  The main qualitative feature that distinguish $\psi^{(\pm)}_{\pm}$ from Weyl plane waves is that their spin plane 

  \begin{equation}
    \rho S = \psi I \sigma_3 \widetilde \psi = (\pm) \rho  R I \sigma_3 \widetilde R = (\pm) \rho R_0 I \sigma_3 \widetilde R_0
  \end{equation}

  does \emph{not} vanish, and the vectors ($i = 1, 2$)

  \begin{equation}
    e_i = R \gamma_i \widetilde R = R_0 \gamma_i \widetilde R_0 e^{\mp S 2 p \cdot x}
  \end{equation} 

  rotate at a frequency $2 \omega$ in this plane, analogous to the zitter of Dirac theory, which describes rotation in the spin plane at a frequency $2 m_e c^2 / \hbar$ (where $m_e$ is the electron mass), and identically to the rotation of electric and magnetic fields of circularly polarized electromagnetic waves.

  This analogy to electromagnetic waves can be made precise. Left and right circularly polarized electromagnetic waves have the form\cite{gap}

    \begin{equation}
      F_\pm = k n \alpha e^{\mp I k \cdot x} = A k n e^{\pm I \hat k (k \cdot x + c)},\label{eq:emwaves}
    \end{equation}

    where $n$ is a constant vector perpendicular to $k$ (i.e. $k \cdot n= 0$), $\alpha = A e^{\mp I c}$ is a constant amplitude and duality transformation, and $\hat k = k \wedge \gamma_0 / |k \wedge \gamma_0|$. As noted by \cite{gap}, positive and negative helicities correspond to left and right handedness respectively. The plus and minus in $F_\pm$ refer to positive and negative helicity, rather than handedness.

    Making the identifications $p = \hbar k$, $R_0 \gamma_1 \widetilde R_0 = n$, and $\rho = \pm A$, then $e_1$ can be seen to rotate identically to the factor $n e^{\pm I \hat k (k \cdot x + c)}$ with twice the frequency and the corresponding electromagnetic field can be placed in the form

    \begin{equation}
      F_\psi \equiv \psi p_0 \gamma_1 \widetilde \psi = \rho p e_1 e^{I \beta}, \label{eq:emfield}
    \end{equation}

    which in this case is

    \begin{equation}
      F_\psi = \pm \rho p e_1.
    \end{equation}

    In this setting, the $e^{I \beta}$ factor is less mysterious than it is in Dirac theory, as it is simply a duality rotation of the elecromagnetic field. In fact, since $p$ is null here, it simplifies to an ordinary spatial rotation of $e_1$:

    \begin{equation}
      F_\psi = \rho p e_1 e^{-S \beta}.
    \end{equation}

    $F_\psi$ is a proper electromagnetic field satisfying

    \begin{equation}
      \nabla F_\psi = 0 \text { and } F_\psi^2 = 0
    \end{equation}

    which are the usual specifications of circularly polarized electromagnetic fields.

    In fact, it will be shown in the next section that $F_\psi$ as given by Equation \ref{eq:emfield} satisfies Maxwell's equations for all massless \emph{and} massive solutions $\psi$ to Equation \ref{eq:dirac} (although, with a disclaimer in the massive case). Furthermore, when $A \not= 0$ in the massless case, $A$ plays the same role as the gauge field that arises in a gauging duality symmetry of Maxwell's equations.

    The only discrepancy is that $e_1$ rotates with twice the frequency of the corresponding electromagnetic wave, so it may be worth making the replacement $p_0 \mapsto \frac{1}{2} p_0$ and taking the equation

    \begin{equation}
      \nabla \psi I \sigma_3 = \frac{1}{2} \psi p_0,\label{eq:free-half}
    \end{equation}

    as primary, so that the frequency of rotation of $e_1$ is given by $\omega$. This amounts to treating spinor solutions as parameterizations of an electromagnetic field $F_\psi$. Doing this will formalize David Hestenes's identification of the zitter rotation of Dirac theory with circulating charge and raises complications with such an identification.

    \section{An Electromagnetic Bilinear Covariant}\label{electromagnetism}

    Suppose $\psi \in G_{1,3}^+$ satisfies Equation \ref{eq:free-half}, taking $p_0$ to be arbitrary and $F_\psi$ is given by Equation \ref{eq:emfield}. Then

    \begin{align}
      \nabla F_\psi &= J_\psi\\
      &= \nabla (\psi p_0 \gamma_1 \widetilde \psi)\\
      &= \nabla \psi p_0 \gamma_1 \widetilde \psi + \dot \nabla \psi p_0 \gamma_1 \dot{\widetilde \psi} \\
      &= \rho p^2 e_2.
    \end{align}

    The current term vanishes for massless solutions precisely because $p^2 = 0$ and does not vanish for massive solutions because $p^2 \not= 0$. Of course, $\nabla \cdot J_\psi = 0$, because $\nabla \wedge F_\psi = 0$.

    When $p^2 \not= 0$, $J_\psi$ describes a circulating electrical current density, which is consistent with Hestenes' identifcation of zitter with circulation of charge. However, the fact that this current is spacelike means that, under the usual interpretation of a current, it propagates at superluminal speeds, compromising its physical integrity.

    In the presence of an electromagnetic gauge field, if $\psi$ satisfies

    \begin{equation}
      \nabla \psi I \sigma_3 - \frac{1}{2} e A \psi = \frac{1}{2} \psi p_0,
    \end{equation}

    then

    \begin{align}
      \nabla F_\psi - e A F_\psi e_2 e_1
      &= \rho p^2 e_2.\label{eq:gaugedem}
    \end{align}

    In the massless case, $A F_\psi e_2 e_1 = \pm I A F_\psi$ and $p^2 = 0$. This yields the equation 

    \begin{equation}
      \nabla F_\psi \mp e I A F_\psi = 0,\label{eq:massless-gauged}
    \end{equation}

    which is identical to the gauged Maxwell equations in vacuum,\cite{duality}\cite{malik}\cite{tiwari} and reinforces the suggestion of \cite{duality} to identify the gauge field $A$ with an electromagnetic potential. \footnote{The role of the coupling constant needs to be worked out.}

    The massive analog of Equation \ref{eq:massless-gauged} is \cite{duality} \cite{tiwari}

    \begin{equation}
      \nabla F - e I A F = J,\label{eq:gauged-duality}
    \end{equation}

    but this is different from Equation \ref{eq:gaugedem}, which possesses the notable distinction that $J_\psi = \rho p^2 e_2$ is invariant under gauge transformations $\psi \mapsto \psi e^{I \sigma_3 \phi}$, unlike Equation \ref{eq:gauged-duality} which transforms under $F \mapsto F e^{I \phi}$ and $J \mapsto J e^{I \phi}$.

  Lastly, a distinction must be made between the currents $J = \psi p_0 \widetilde \psi$ and $J_\psi = \nabla F_\psi = \nabla (\psi p_0 \gamma_1 \widetilde \psi)$. In the massless case, the former describes the lightlike probability current of the fermions described in this paper, whereas the latter vanishes and is associated with a circularly polarized electromagnetic waves given by $F_\psi$. In the massive case, the former describes a timelike probability current and the latter describes a spacelike current associated with $F_\psi$. 

  It should be noted that there exists an antiderivative $F$ of $J$ such that $\nabla F = J$. It has been shown in \cite{continuity-equation} that, under suitable boundary conditions, this antiderivative is a bivector (i.e. an electromagnetic field, when suited with the appropriate units).

  \section{Conclusion}

  It would be interesting to know whether the massless particles described here are compatible with Hestenes' model of the electron as a lightlike particle with electromagnetic interactions, trapped in helical motion.\cite{zitter} If so, perhaps the gauge field provides a constraint force.

  Furthermore, the zitter mechanism that motivated Hestenes to identify the electron with a circulating current would suggest, by the same reasoning, that massless solutions are intimately related to electromagnetic waves. Although, a direct identification runs straight into the Pauli exclusion principle.

  \newpage

  \appendix

  \section{Matrix Formulation} \label{matrix}

  Following Equation 8.70 in \cite{gap}, Equations \ref{eq:even} and \ref{eq:odd} take the following form in matrix representation.

  \begin{equation}
    \hat \gamma^\mu (i \partial_\mu - e A_\mu) | \psi_+ \rangle = \omega_0 (1 + \hat \gamma_5) | \psi_+ \rangle
  \end{equation}

  and

  \begin{equation}
    \hat \gamma^\mu (i \partial_\mu - e A_\mu) | \psi_- \rangle = \omega_0 (1 - \hat \gamma_5) | \psi_- \rangle,
  \end{equation}

  where

  \begin{align}
    \hat \gamma^0 = \begin{pmatrix} 1 & 0 \\ 0 & -1 \end{pmatrix},\quad \hat \gamma^k = \begin{pmatrix} 0 & \hat \sigma^k \\ -\hat \sigma^k & 0 \end{pmatrix},\quad \hat \gamma_5 = \begin{pmatrix} 0 & 1 \\ 1 & 0 \end{pmatrix}.
  \end{align}

  See \cite{gap} for more details on the isomorphism.

  \begin{thebibliography}{9}
    \bibitem{gap} 
      C. Doran and A. Lasenby.
      \emph{Geometric Algebra for Physicists}. Cambridge University Press (2003).

    \bibitem{hestenes}
      D. Hestenes.
      \emph{The Zitterbewegung Interpretation of Quantum Mechanics}.
      Found. Physics. 20 (1990).

    \bibitem{duality}
      L. Burns.
      \emph{Gauging Duality Symmetry}.
      \url{https://github.com/lukeburns/gauge-duality}. (Work in Progress).

    \bibitem{malik}
      R. P. Malik and T. Pradhan 
      \emph{Local Duality Invariance of Maxwell's Equations}.
      Z. Phys. C - Particles and Fields 28 (1985).

    \bibitem{tiwari}
      S.C. Tiwari.
      \emph{Axion electrodynamics in the duality perspective}
      Modern Physics Letters A.
      Vol. 30, No. 40 (2015).

    \bibitem{continuity-equation}
      L. Burns.
      \emph{The Continuity Equation Implies Maxwell's Equations}.
      \url{https://github.com/lukeburns/maxwells-equations}. (Work in Progress).

    \bibitem{zitter}
      D. Hestenes.
      \emph{Zitterbewegung in Quantum Mechanics --- a research program}.
      \url{https://arxiv.org/abs/0802.2728}.

  \end{thebibliography}

\end{document}
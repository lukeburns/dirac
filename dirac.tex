\documentclass{article}
\usepackage[margin=1.25in]{geometry}
\usepackage{amsmath}
\usepackage{amsfonts}

\title{An Extension of the Dirac Equation}
\author{Luke Burns}

\begin{document}
  \maketitle

  \section{Introduction}

  The free Dirac equation in STA is

  \begin{equation}
    \nabla \psi I \sigma_3 = \psi p_0, \label{eq:dirac}
  \end{equation}

  where

  \begin{equation}
    p_0 = m \gamma_0. \label{eq:mass}
  \end{equation} 

  A less restrictive constraint is just that

  \begin{equation}
    p_0^2 = m^2
  \end{equation}

  be constant. With this lightened restriction, Equation \ref{eq:dirac} admits a new class of null solutions, where $p_0^2 = 0$ but $\psi p_0 \not = 0$.

  \section{General Results}

\subsection{Constraints on $p_0$}

  The extension of the full Dirac equation, including the electromagnetic gauge field

  \begin{equation}
    \nabla \psi I \sigma_3 - e A \psi = \psi p_0 \label{eq:full}
  \end{equation}

  is only invariant under the usual replacements $\psi \mapsto \psi e^{\alpha I \sigma_3}$ and $A \mapsto A - \nabla \alpha$ if $p_0 \cdot I\sigma_3 = 0$:

  \begin{align}
    \nabla (\psi e^{\alpha I \sigma_3}) I \sigma_3 - e (A - \nabla \alpha) \psi e^{\alpha I \sigma_3} &= (\nabla \psi I \sigma_3  - e A \psi) e^{\alpha I \sigma_3} \\
     &= \psi p_0 e^{\alpha I \sigma_3} \\
     &= \psi e^{\alpha I \sigma_3} p_0.
  \end{align}

  This is an essential feature of Dirac theory, so we'll restrict $p_0$ preserve it. We require that

  \begin{equation}
    p_0 = E \gamma_0 \pm |\vec p| \gamma_3 = (E + \vec p) \gamma_0.
  \end{equation} 

  If $p_0^2 = m^2 \not= 0$, then 

  \begin{equation}
    p_0 = R m \gamma_0 \widetilde R
  \end{equation} 

   for $R = e^{\gamma_3 \gamma_0 \alpha/2}$ with $\alpha$ given by $\tanh(\alpha) = |\vec p|/E$. Then $p_0 \mapsto \widetilde R p_0 R$ reduces to Equation \ref{eq:mass}.

  On the other hand, if $p_0^2 = 0$, then

  \begin{equation}
    p_0 = \omega (1 + \hat p) \gamma_0, \label{eq:massless}
  \end{equation}

  for $\omega = E = |\vec p|$, which gives an equation of the form

  \begin{equation}
    \nabla \psi I \sigma_3 = \omega \psi (1 \pm \sigma_3) \gamma_0.\label{eq:extension}
  \end{equation}

  % Note that there are two classes of null solutions to this equation. Those for which the direction of the momentum density $\psi \hat p \gamma_0 \widetilde \psi = \pm \psi \gamma_3 \widetilde \psi$ is (1) parallel and (2) anti-parallel with the spin vector $s = \psi \gamma_3 \widetilde \psi$.

  Solutions to Equation \ref{eq:dirac} satisfying Equation \ref{eq:massless} are distinct from those satisfying Equation \ref{eq:mass}. In particular, these solutions are distinct from the massless solutions to Equation \ref{eq:mass}, because they yield different observables. This will be demonstrated in the next section.

  \subsection{The Conserved Vector Current}

  In general, if $v_0$ is a constant vector, then

  \begin{equation}
    \nabla \cdot (\psi v_0 \widetilde \psi) = \langle v_0 \wedge p_0 (I \sigma_3 \psi \widetilde \psi) \rangle + \langle v_0 \cdot I \sigma_3 (\widetilde \psi e A \psi) \rangle,
  \end{equation}

  and

  \begin{equation}
    \nabla \cdot (\psi v_0 \widetilde \psi) = 0 \iff v_0 \wedge p_0 = 0. 
  \end{equation}

  That is, energy-momentum density $\rho p = \psi p_0 \widetilde \psi$ is the only vector-valued bilinear covariant conserved generally (up to a constant multiple).

  In particular, notice that the conservation of the usual probability current $J = \psi \gamma_0 \widetilde \psi = \rho v$ does \emph{not} vanish for $p_0 = \omega (1 + \hat p) \gamma_0$, since $\gamma_0 \wedge p_0 = - \vec p = \mp \omega \sigma_3$:

  \begin{align}
    \nabla \cdot J &= \langle \gamma_0 \wedge p_0 (I \sigma_3 \psi \widetilde \psi) \rangle \\
                   &= \mp \omega \rho \langle I e^{I \beta} \rangle \\
                   &= \pm \omega \rho \sin \beta.
  \end{align}

  A consequence of this result is that the usual interpretation of $J$ as a probability current does not extend to Equation \ref{eq:massless}. If solutions to this equation prove useful, then producing an interpretation that applies to both Equation \ref{eq:mass} and Equation \ref{eq:massless} may be worthwhile.

  Is there an obvious route to a probabilistic interpretation that works for general $p_0$?

  % \subsection{Possible Interpretations}

  % Is there an obvious route to a probabilistic interpretation that works for general $p_0$?

  % Current strategy: toy with the massless theory.

  % \section{The Massless Theory}

  % Equation \ref{eq:dirac} with null $p_0$ is

  % \begin{equation}
  %   \nabla \psi I \sigma_3 = \omega \psi (1 \pm \sigma_3) \gamma_0
  % \end{equation}

  % or equivalently

  % \begin{equation}
  %   \nabla \psi = \mp I \omega \psi (1 \pm \sigma_3) \gamma_0,
  % \end{equation}

  % since $(1 \pm \sigma_3) (\pm \sigma_3) = (1 \pm \sigma_3)$. If there are constraints on $A$ (WIP next section), then it'll be worth considering this generalization to admit arbitrary $A$. Furthermore, for electromagnetic waves, duality rotations and rotations in the $E \wedge B$ plane are equivalent.

  % Equation \ref{eq:full} with null $p_0$ can be written

  % \begin{equation}
  %   \nabla \psi I \sigma_3 - e A \psi = \omega \psi (1 \pm \sigma_3) \gamma_0
  % \end{equation}

  % and has a number of notable properties.

  % Firstly, $p_0^2 = 0$ implies that $p^2 = (R p_0 \widetilde R)^2 = 0$, so this is the momentum of a massless particle. Secondly, $p_0 \cdot (I \sigma_3) = 0$ implies that $p = R p_0 \widetilde R$ is always perpendicular to the spin plane $S = \psi I \sigma_3 \widetilde \psi$. These two features highly suggest that Equation \ref{eq:dirac} with Equation \ref{eq:photon} describes photons.\footnote{How can I be sure that this doesn't describe a spin-1/2 particle? There is a factor of 1/2 tied up in the fact that $p_0/2\omega$ is idempotent and $m\gamma_0/m$ is idempotent. Does this have anything to do with spin?}

  % \subsection{Constraints on $A$}

  % $\nabla \cdot (\rho p) = 0$ implies that

  % \begin{align}
  %   \nabla (\rho p) &= \nabla \wedge (\rho p)\\
  %   &= 2 e A p s
  % \end{align}

  % implies that...

  % \subsection{Relation to Electromagnetic Fields}

  % To take the possibility that these solutions describe photons, consider the field\footnote{This is similar to Vaz and Rodriguez's approach: https://arxiv.org/pdf/hep-th/9511181v1.pdf.}

  % \begin{equation}
  %   F = \nabla \psi I \sigma_3 n_0 \widetilde \psi = \psi p_0 n_0 \widetilde \psi \label{eq:faraday}
  % \end{equation}

  % where $n_0 \cdot p_0 = 0$, which satisfies

  % \begin{align}
  %   F^2 &= \psi p_0 n_0 \widetilde \psi \psi p_0 n_0 \widetilde \psi \\
  %       &= - \rho e^{I\beta} \psi n_0 p_0^2 n_0 \widetilde \psi \\
  %       &= 0.
  % \end{align}

  % % and

  % % \begin{align}
  % %   \nabla F &= \nabla (\nabla \psi I \sigma_3 n_0 \widetilde \psi) \\
  % %    &= (\nabla^2 \psi) I \sigma_3 n_0 \widetilde \psi + \dot \nabla(\psi p_0 n_0 \dot{\widetilde{\psi}}) \\
  % %    &= \gamma^\mu (\psi p_0 n_0 p_\mu \Omega_\mu \widetilde \psi)\\
  % %    &= \psi p_0^2 n_0 \Omega \widetilde \psi \\
  % %    &= 0.
  % % \end{align}

  % % Hold up is this right? Not sure this sum is good.

  % How does $F$ behave under gauge transformations?

  % Furthermore, $p_0 \cdot \gamma_3 = -\omega$ implies

  % \begin{equation}
  %   s \cdot p = - \omega \rho,
  % \end{equation}

  % where $s = \psi \gamma_3 \widetilde \psi$, so that

  % \begin{equation}
  %   \nabla \cdot J = - 2 \omega \rho \sin (\beta) \not = 0.
  % \end{equation}

  % What does $J$ mean in this context?

  % \subsection{Plane Waves}

  % Definition (13) was selected in order to reduce to the general form of left or right circularly polarized electromagnetic waves in the plane wave case, where $\psi = \psi_0 e^{\pm I p \cdot x/2}$, $\psi_0 = \rho^{1/2} e^{I \beta/2} R$ constant, and the added constraint $p = R p_0 \widetilde R$:\footnote{GAP 7.142. Is this constraint necessary? Would it make sense to use $e^{\pm I p_0 \cdot x}$?}

  % \begin{align}
  %   F &= \psi p_0 n_0 \widetilde \psi \\
  %     &= pn \rho e^{I \beta}e^{\pm I p \cdot x}.
  % \end{align}

  % \subsection{Interactions}

  % What if $A$ is a potential for an electromagnetic wave?

  % \section{Scratch work}

  % \subsection{$\nabla \cdot J$}

  %   \begin{align}
  %   \nabla \psi &= - (e A \psi + \psi p_0) I \sigma_3 \\
  %   \nabla \psi \gamma_0 \widetilde \psi &= - (e A \psi + \psi p_0) I \sigma_3 \gamma_0 \widetilde \psi \\
  %   &= - I (e A \psi + \psi p_0) \gamma_3 \widetilde \psi \\
  %   &= - I (e A + p e^{-I \beta}) s \\
  %   \langle \nabla \psi \gamma_0 \widetilde \psi \rangle &= - \langle I (e A + p e^{-I \beta}) s \rangle \\
  %   &= \langle I s e A \rangle +  \langle I s p e^{-I \beta}) \rangle
  % \end{align}

  % \begin{align}
  %   \dot{\widetilde{\psi}} \dot \nabla &= - (e \widetilde{A \psi I \sigma_3} + \widetilde{\psi p_0 I \sigma_3})\\
  %   &= e I \sigma_3 \widetilde{\psi} A + I \sigma_3 p_0 \widetilde{\psi} \\
  %   \psi \gamma_0 \dot{\widetilde{\psi}} \dot \nabla &= \psi \gamma_0 I \sigma_3 \widetilde{\psi} e A + \psi \gamma_0 I \sigma_3 p_0 \widetilde{\psi} \\
  %   &= I (\psi \gamma_3 \widetilde{\psi} e A + \psi \gamma_3 p_0 \widetilde{\psi})\\
  %   &= I (s e A + s \widetilde{\psi p_0 \psi^{-1}})\\
  %   &= Is (e A + p e^{-I \beta})\\
  %   \langle \psi \gamma_0 \dot{\widetilde{\psi}} \dot \nabla \rangle &= \langle I seA \rangle + \langle I s p e^{-I \beta} \rangle
  % \end{align}

  % where $p e^{-I \beta} = \psi p_0 \psi^{-1}$.

  % \subsubsection{$\widetilde{\psi^{-1}} = \widetilde{\psi}^{-1}$}

  % $\widetilde{\psi^{-1}} = \widetilde{\rho^{-1/2} e^{-I \beta/2} \widetilde R} = \rho^{-1/2} e^{-I \beta/2} R$

  % $\widetilde{\psi}^{-1} = (\rho^{1/2} e^{I \beta/2} \widetilde R)^{-1/2} = \rho^{-1/2} e^{-I \beta/2} R$

  % \subsection{Observables for $p_0 = \omega (1 \pm \sigma_3)\gamma_0$}

  % The momentum relates directly to the current density $J = \psi \gamma_0 \widetilde \psi $, the spin vector $s = \psi \gamma_3 \widetilde \psi$, and the density / duality term $\psi \widetilde \psi = \rho e^{I \beta}$ in the following way

  % \begin{align}
  %   p &= R p_0 \widetilde R \\
  %     & = \omega R (\gamma_0 \pm \gamma_3) \widetilde R \\
  %     &= \omega (J + s)/\rho.
  % \end{align}

  % And the spin plane is similarly related to these three observables.

  % \begin{align}
  %   S &= \psi I \sigma_3 \widetilde \psi \\
  %     &= I \psi \gamma_3 \gamma_0 \widetilde \psi \\
  %     &= I s J e^{I \beta} / \rho.
  % \end{align}

  % \subsection{Alternate definition of $F$}

  % How does

  % \begin{equation}
  %   F = \nabla (\psi I \sigma_3 n \widetilde \psi)
  % \end{equation}

  % compare with Equation \ref{eq:faraday}? Is it still a bivector?

  % \subsection{Duality rotations}

  % Equation \ref{eq:dirac} is equivalent to

  % \begin{equation}
  %   \nabla \psi I = - \omega \psi (1 \pm \sigma_3) \gamma_0,
  % \end{equation}

  % due to the property $(1 \pm \sigma_3) \sigma_3 = 1 \pm \sigma_3$. This means that duality transformations and rotations in the $I \sigma_3$ plane are identical.

  % What are the consequences of this? Does this place constraints on $\psi$? Does this affect the gauging process?

  % No need to specify $\sigma_3$:

  % \begin{equation}
  %   \nabla \psi I = - \psi p_0,
  % \end{equation}

  % for $p_0^2 = 0$.

  % \subsection{Electromagnetic Fields}

  % Consider

  %   \begin{equation}
  %     \widetilde R \nabla R = (e A_0 + p_0 e^{- I \beta}) s_0
  %   \end{equation}

  % where $A_0 = \widetilde R A R$ and $s_0 = - I \sigma_3$.

\end{document}
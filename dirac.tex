\documentclass{article}
\usepackage[margin=1.25in]{geometry}
\usepackage{amsmath}
\usepackage{amsfonts}

\title{An Extension of the Dirac Equation}
\author{Luke Burns}

\begin{document}
  \maketitle

  \section{Introduction}

  The free Dirac equation in STA is

  \begin{equation}
    \nabla \psi I \sigma_3 = \psi p_0, \label{eq:dirac}
  \end{equation}

  where

  \begin{equation}
    p_0 = m \gamma_0. \label{eq:mass}
  \end{equation} 

  A less restrictive constraint is just that

  \begin{equation}
    p_0^2 = m^2
  \end{equation}

  be constant. With this lightened restriction, Equation \ref{eq:dirac} admits a new class of null solutions, where $p_0^2 = 0$ but $\psi p_0 \not = 0$.

  \section{General Results}

\subsection{Constraints on $p_0$}

  The extension of the full Dirac equation, including the electromagnetic gauge field

  \begin{equation}
    \nabla \psi I \sigma_3 - e A \psi = \psi p_0 \label{eq:full}
  \end{equation}

  is only invariant under the usual replacements $\psi \mapsto \psi e^{\alpha I \sigma_3}$ and $A \mapsto A - \nabla \alpha$ if $p_0 \cdot I\sigma_3 = 0$:

  \begin{align}
    \nabla (\psi e^{\alpha I \sigma_3}) I \sigma_3 - e (A - \nabla \alpha) \psi e^{\alpha I \sigma_3} &= (\nabla \psi I \sigma_3  - e A \psi) e^{\alpha I \sigma_3} \\
     &= \psi p_0 e^{\alpha I \sigma_3} \\
     &= \psi e^{\alpha I \sigma_3} p_0.
  \end{align}

  This is an essential feature of Dirac theory, so we'll restrict $p_0$ preserve it. We require that

  \begin{equation}
    p_0 = E \gamma_0 \pm |\vec p| \gamma_3 = (E + \vec p) \gamma_0.
  \end{equation} 

  If $p_0^2 = m^2 \not= 0$, then 

  \begin{equation}
    p_0 = R m \gamma_0 \widetilde R
  \end{equation} 

   for $R = e^{\gamma_3 \gamma_0 \alpha/2}$ with $\alpha$ given by $\tanh(\alpha) = |\vec p|/E$. Then $p_0 \mapsto \widetilde R p_0 R$ reduces to Equation \ref{eq:mass}.

  On the other hand, if $p_0^2 = 0$, then

  \begin{equation}
    p_0 = \omega (1 + \hat p) \gamma_0, \label{eq:massless}
  \end{equation}

  for $\omega = E = |\vec p|$. 

  Note that there are two classes of null solutions. Those for which the direction of the momentum density $\psi \hat p \gamma_0 \widetilde \psi = \pm \psi \gamma_3 \widetilde \psi$ is (1) parallel and (2) anti-parallel with the spin vector $s = \psi \gamma_3 \widetilde \psi$.

  Solutions to Equation \ref{eq:dirac} satisfying Equation \ref{eq:massless} are distinct from those satisfying Equation \ref{eq:mass}. In particular, these solutions are distinct from the massless solutions to Equation \ref{eq:mass}, because they yield different observables. This will be demonstrated in the next section.

  \subsection{The Conserved Vector Current}

  In general, if $v$ is a constant vector, then

  \begin{equation}
    \nabla \cdot (\psi v \widetilde \psi) = \langle v \wedge p_0 (I \sigma_3 \psi \widetilde \psi) \rangle + \langle v \cdot I \sigma_3 (\widetilde \psi e A \psi) \rangle.
  \end{equation}

  Notice that the conservation of the usual probability current $J = \psi \gamma_0 \widetilde \psi = \rho v$ does \emph{not} vanish for $p_0 = \omega (1 + \hat p) \gamma_0$, since $\gamma_0 \wedge p_0 = - \vec p = \mp \omega \sigma_3$:

  \begin{align}
    \nabla \cdot J &= \langle \gamma_0 \wedge p_0 (I \sigma_3 \psi \widetilde \psi) \rangle \\
                   &= \mp \omega \rho \langle I e^{I \beta} \rangle \\
                   &= \pm \omega \rho \sin \beta.
  \end{align}

  A consequence of this result is that the usual interpretation of $J$ as a probability current does not extend to Equation \ref{eq:massless}. If solutions to this equation prove useful, then producing an interpretation that applies to both Equation \ref{eq:mass} and Equation \ref{eq:massless} may be worthwhile.

  In general, we have

  \begin{equation}
    \nabla \cdot (\psi v \widetilde \psi) = 0 \iff v \wedge p_0 = 0. 
  \end{equation}

  That is, energy-momentum density $\rho p = \psi p_0 \widetilde \psi$ is the only vector-valued bilinear covariant conserved in general.

  Is there an obvious route to a probabilistic interpretation that works for general $p_0$?

\end{document}
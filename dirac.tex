\documentclass{article}
\usepackage[margin=1.25in]{geometry}
\usepackage{amsmath}
\usepackage{amsfonts}
\usepackage{multicol}
\usepackage{hyperref}

\title{An Extension of the Dirac Equation}
\author{Luke Burns}

\begin{document}
  \maketitle

  \section{Introduction}

  In this paper, I will show that a minimal extension of the Dirac equation admits a description of massless, electrically charged, spin-1/2 particles that violate charge conjugation and time reversal (CP) symmetries.

  The free Dirac equation as expressed in the Space Time Algebra (STA) is\cite{gap}

  \begin{equation}
    \nabla \psi I \sigma_3 = \psi p_0, \label{eq:dirac}
  \end{equation}

  where

  \begin{equation}
    p_0 = m \gamma_0. \label{eq:mass}
  \end{equation} 

  This equation in part describes the dynamics of a spinor $\psi = (\rho e^{I \beta})^{1/2}R$ that rotates, boosts, and dilates the momentum of a particle in its rest frame $p_0 = m \gamma_0$ onto a probability current $mJ = \psi p_0 \widetilde \psi = m \rho e_0$ where $e_0 = R \gamma_0 \widetilde R$. However, this description fails for massless particles, since they have no rest frame.

  A minimal extension to this equation that accomodates a similar description for massless particles is to simply require that $p_0$ be constant, so that $\psi$ describes the dynamics of a spinor that rotates, boosts, and dilates the momentum $p_0$ of a particle in some arbitrary ``initial'' frame onto a probability current $\psi p_0 \widetilde \psi$ (it will be shown that this works in general in Section \ref{probability}).

  This extension including the electromagnetic gauge field can be written

  \begin{equation}
    \nabla \psi I \sigma_3 - e A \psi = \psi p_0. \label{eq:full}
  \end{equation}

  In Section \ref{constraints}, we'll work out the physical constraints on $p_0$ and show that Equation \ref{eq:full} admits the usual solutions to the Dirac equation and a new class of null solutions, where $p_0^2 = 0$ but $\psi p_0 \not = 0$. Then we'll work out the symmetries of Equation \ref{eq:full} in Section \ref{symmetries}. Finally, we will determine the observables of solutions in Section \ref{probability} and present plane wave solutions in Section \ref{waves}. 

  The Matrix formulation is tucked away in Appendix \ref{matrix}. I encourage the reader to check out STA, which provides the dynamical spinor description used above to motivate the extension. See \url{https://github.com/ga/Resources} for a collection of resources.

  \section{Constraints}\label{constraints}

  Equation \ref{eq:full} is only invariant under the usual replacements $\psi \mapsto \psi e^{\alpha I \sigma_3}$ and $A \mapsto A - \nabla \alpha$ if 

  \begin{equation}
    p_0 I\sigma_3 = I \sigma_3 p_0. \label{momentum-perp-spin}
  \end{equation} 

  That is, $p_0$ is perpendicular to the plane $I \sigma_3$. This is an essential feature of Dirac theory, so we'll restrict $p_0$ to preserve it. We require that

  \begin{equation}
    p_0 = E \gamma_0 \pm |\vec p| \gamma_3.
  \end{equation} 

  Additionally, if solutions are to satisfy the Klein-Gordon equation, we must have 

  \begin{equation}
    p_0^2 \geq 0.
  \end{equation}

  If $p_0^2 > 0$, then 

  \begin{equation}
    p_0 = R m \gamma_0 \widetilde R\label{eq:R}
  \end{equation} 

   for $R = e^{\gamma_3 \gamma_0 \alpha/2}$ with $\alpha$ given by $\tanh(\alpha) = \pm |\vec p|/E$. 

   If $\psi$ is a solution to Equation \ref{eq:full} and $p_0^2 > 0$, then $\psi R$ is a solution to the Dirac equation. On the other hand, if $\psi$ is instead a solution to the Dirac equation, then $\psi \widetilde R$ is a solution to Equation \ref{eq:full}.\footnote{We will see in Section \ref{symmetries} that $\psi$ is in general a multivector, so if $\psi$ is a solution to Equation \ref{eq:full}, then its even and odd parts satisfy Equation \ref{eq:full} independently. The argument above can be made for each of these solutions independently using Equations \ref{eq:even} and \ref{eq:odd}.} That is, solutions to Equation \ref{eq:full} are in one-to-one correspondence to solutions of the Dirac equation.

   On the other hand, if $p_0^2 = 0$, then

  \begin{equation}
    p_0 = \omega_0 (1 \pm \sigma_3) \gamma_0, \label{eq:massless}
  \end{equation}

  for $\omega_0 = E = |\vec p|$. Note that $\psi$ can be decomposed into 

  \begin{equation}
    \psi = \psi \frac{1 + \sigma_3}{2} + \psi \frac{1 - \sigma_3}{2}.
  \end{equation}

  If $\psi$ is a solution to Equation \ref{eq:full} and $p_0^2 = 0$, then the projection $\psi \frac{1 \mp \sigma_3}{2}$ gives a solution to the (massless) Dirac equation, since $\psi \frac{1 \mp \sigma_3}{2} p_0 = 0$. However, this is not invertible. The only way to recover $\psi$ would be from $\psi \frac{1 \pm \sigma_3}{2}$. The problem is that this is another solution to Equation \ref{eq:full} and \emph{not} a solution to the Dirac equation, since $\psi \frac{1 \pm \sigma_3}{2} p_0 = 2 \psi p_0$.\footnote{Makes me want to throw in a factor of $1/2$ on the RHS of Equation \ref{eq:full}. Another reason for doing this arises in Section \ref{waves} is that $\omega_0$ is actually twice the frequency of rotation generated by $\psi$ in the initial frame.} So there is no way to recover general solutions to Equation \ref{eq:full} from solutions to the massless Dirac equation. 

  This means that Equation \ref{eq:full} contains null solutions that are distinct from solutions to the massless Dirac equation and, furthermore, that these are the \emph{only} new solutions admitted by the extension. In this sense, the extension is minimal.

  \section{Symmetries} \label{symmetries}

  Unlike the Dirac equation, Equation \ref{eq:full} is not invariant under

  \begin{equation}
    \psi \mapsto \psi \gamma_0.\label{eq:g0conjugation}
  \end{equation}

  If $\psi$ is a solution to

  \begin{equation}
    \nabla \psi I \sigma_3 - e A \psi = \psi p_0,\label{eq:plus}
  \end{equation}

  then $\psi' = \psi \gamma_0$ is a solution to

  \begin{equation}
    \nabla \psi' I \sigma_3 - e A \psi' = \psi' \overline p_0 \gamma_0, \label{eq:minus}
  \end{equation}

  where $\overline M \equiv \gamma_0M\gamma_0$ is minus the reflection of any multivector $M$ across the $\gamma_0$ axis (i.e. $\overline{E \gamma_0 + |\vec p|\gamma_3} = E \gamma_0 - |\vec p|\gamma_3$).

  That is, Equation \ref{eq:full} distinguishes between even and odd fields, which is a key reason we are not able to find a one-to-one correspondence between massless solutions to Equation \ref{eq:full} and the Dirac equation. In general, $\psi$ is a full multivector and can be decomposed into

  \begin{equation}
    \psi = \langle \psi \rangle_+ + \langle \psi \rangle_-,
  \end{equation}

  where $\langle \psi \rangle_+$ and $\langle \psi \rangle_-$ are even and odd multivectors respectively that are independent solutions to

  \begin{equation}
    \nabla \langle \psi \rangle_\pm I \sigma_3 - e A \langle \psi \rangle_\pm = \langle \psi \rangle_\pm p_0.\label{eq:decoupled}
  \end{equation}

  We can therefore make the choice of $p_0 = \omega_0 (1 + \sigma_3) \gamma_0$ for the massless case without loss of generality, because if $\psi_+ = \langle \psi \rangle_+$ is a solution to

  \begin{equation}
    \nabla \psi_+ I \sigma_3 - e A \psi_+ = \omega_0 \psi_+ (1 + \sigma_3) \gamma_0,\label{eq:even}
  \end{equation}

  then the even multivector 

  \begin{equation}
    \psi_- = \langle \psi \rangle_- \gamma_0\label{eq:0decomposition}
  \end{equation} 

  is a solution to

  \begin{equation}
    \nabla \psi_- I \sigma_3 - e A \psi_- = \omega_0 \psi_- (1 - \sigma_3) \gamma_0.\label{eq:odd}
  \end{equation}

  At first glance, it appears that $\psi_+$ and $\psi_-$ have opposite helicity. However, this is not exactly the case. For instance, if $\psi_+$ is a solution to Equation \ref{eq:even}, then $\psi_+ \sigma_1$ is a solution to the same equation with opposite charge and helicity. The precise difference between $\langle \psi \rangle_+$ and $\langle \psi \rangle_-$ is the combination of charge and helicity, which you can see by inspecting Equations \ref{eq:even} and \ref{eq:odd}. 

  We can make better sense of this by looking at charge, parity, and time reversal conjugations. It turns out that there is no grade preserving map between solutions to Equation \ref{eq:even} and Equation \ref{eq:odd} in the presence of a gauge field. As consequence, charge conjugation and time reversal symmetries are violated. These conjugations are given by

  \begin{align}
    \hat C \psi : \psi \mapsto \psi \gamma_1 &\iff eA \mapsto - eA \label{eq:charge}\\
    \hat P \psi : \psi(x) \mapsto \overline \psi(\overline x) &\iff \nabla \mapsto \overline \nabla, eA \mapsto e\overline A, \text{ and } p_0 \mapsto \overline p_0 \label{eq:parity}\\
    \hat T \psi : \psi(x) \mapsto I \overline \psi(-\overline x) \gamma_1 &\iff \nabla \mapsto -\overline\nabla, I\sigma_3 \mapsto - I\sigma_3, eA \mapsto -e\overline A, \text{ and } p_0 \mapsto - \overline p_0.\label{eq:time}
  \end{align}

  Parity is grade preserving, and of course, so is the combined CPT conjugation, given by

  \begin{equation}
    \psi(x) \mapsto I \psi(-x) \iff I \sigma_3 \mapsto - I \sigma_3 \text{ and } A \mapsto - A. \label{eq:cpt}
  \end{equation}

  However, charge and time reversal conjugations are not. This means that charge and time reversal symmetries are violated by Equations \ref{eq:even} and \ref{eq:odd}.

  The reason that charge and time reversal conjugations are grade preserving in Dirac theory is that Equation \ref{eq:g0conjugation} leaves the Dirac equation invariant. The situation here is different. There is no odd-valued conjugation that leaves Equations \ref{eq:even} and \ref{eq:odd} invariant in general that would enable us to construct grade preserving charge and time reversal conjugations out of Equations \ref{eq:charge} and \ref{eq:time}, which one can confirm with a glance at the possible vector-valued conjugations:

  \begin{align}
    \psi \mapsto \psi \gamma_0 &\iff p_0 \mapsto \overline p_0 \\
    \psi \mapsto \psi \gamma_{1,2} &\iff e A \mapsto - eA \\
    \psi \mapsto \psi \gamma_3 &\iff p_0 \mapsto - \overline p_0\\
    \psi \mapsto \gamma_0 \psi &\iff \nabla \mapsto \overline \nabla \text{ and } eA \mapsto - e \overline A.
  \end{align}

  When the gauge field vanishes, Equation \ref{eq:charge} is an odd-valued conjugation that leaves Equation \ref{eq:full} unchanged, and so

  \begin{equation}
    \psi \mapsto \psi \sigma_1,
  \end{equation}

  provides a grade preserving map between solutions to Equations \ref{eq:even} and \ref{eq:odd}.

  \section{The Probability Current}\label{probability}

  Essential to Dirac theory is its probabilistic intepretation, which depends on a conserved probability current $J$ satisfying the continuity equation

  \begin{equation}
    \nabla \cdot J = 0.
  \end{equation}

  The usual probability current $\psi \gamma_0 \widetilde \psi$ of Dirac theory is not conserved here. To see this, consider the following, for a constant vector $v_0$.

  \begin{equation}
    \nabla \cdot (\psi v_0 \widetilde \psi) = \langle v_0 \wedge p_0 (I \sigma_3 \psi \widetilde \psi) \rangle + \langle v_0 \cdot I \sigma_3 (\widetilde \psi e A \psi) \rangle,\label{eq:current-expansion}
  \end{equation}

  which gives a condition for conservation

  \begin{equation}
    \nabla \cdot (\psi v_0 \widetilde \psi) = 0 \iff v_0 \wedge p_0 = 0.
  \end{equation}

  The second term in Equation \ref{eq:current-expansion} vanishes, because $v_0 \wedge p_0 = 0$ implies $v_0 \cdot I \sigma_3 = 0$, due to Equation \ref{momentum-perp-spin}.

  This means that $\psi p_0 \widetilde \psi$ is the only vector-valued bilinear covariant conserved in general (up to a constant multiple). Furthermore, the fact that $\nabla \cdot (\psi p_0 \widetilde \psi) = 0$ implies the existence of streamlines with tangents given by $p = R p_0 \widetilde R$, which are timelike if $p_0^2 > 0$ and lightlike if $p_0^2 = 0$. The usual probability current $\psi \gamma_0 \widetilde \psi$ is not conserved because $\gamma_0 \wedge p_0 \not= 0$.

  The normalization procedure

  \begin{equation}
    \int d^3x \gamma_0 \cdot J = 1\label{eq:mnormal}
  \end{equation}

  can be extended straightforwardly. In Dirac theory, Equation \ref{eq:mnormal} is equivalent to

  \begin{equation}
    \int d^3x \gamma_0 \cdot (\psi p_0 \widetilde \psi) = m,
  \end{equation}

  which simply ensures that integrating energy density (in the $\gamma_0$ frame) over all of space is just the rest energy of the particle.\footnote{Note, there is a certain arbitrariness to this, since it is frame dependent.}

  Since massless particles do not have rest energy, a reasonable generalization of this is

  \begin{equation}
    \int d^3x \gamma_0 \cdot (\psi p_0 \widetilde \psi) = c. \label{eq:normalization}
  \end{equation}

  for a constant $c$. Any choice of $c$ determines a probability current $J = \psi p_0 \widetilde \psi / c$ with a normalized probability density $J_0 = \gamma_0 \cdot J$. Selecting $c = \gamma_0 \cdot p_0$ may be a convenient choice, because it coincides with $\omega_0$ in the massless theory and aligns with the usual choice $m$ in the massive theory.\footnote{This is frame dependent. It would be nice to make this frame independent.}

  \section{Plane Waves} \label{waves}

  Plane wave solutions are given by

  \begin{equation}
    \nabla \psi I \sigma_3 = p \psi,\label{eq:plane}
  \end{equation}

  where $p$ is a constant, which tells us that

  \begin{equation}
    p \psi = \psi p_0.
  \end{equation}

  Using the decomposition $\psi = (\rho e^{I \beta})^{1/2} R$,

  \begin{equation}
    p e^{I \beta} = R p_0 \widetilde R
  \end{equation}

  implies that $e^{I \beta} = \pm 1$, and hence

  \begin{equation}
    p = \pm R p_0 \widetilde R.
  \end{equation}

  $p$ is constant implies that $\rho$ is constant and, given the decomposition $R = R_\parallel R_\perp$, where $R_\perp$ satisfies $p_0 = R_\perp p_0 \widetilde R_\perp$, $R_\parallel$ is constant and $R_\perp = e^{I \sigma_3 \theta(x)}$.

  This implies that 

  \begin{equation}
    \theta = \pm p \cdot x + c(x),
  \end{equation}

  where $c(x)$ satisfies $\nabla c(x) = 0$ (is monogenic).

  Taking $c(x) = 0$, for every $\rho$ and $R_\parallel$, which determine a constant Lorentz transformation and dilation of $p_0$, we have two solutions

  \begin{equation}
    \psi_1 = \rho^{1/2} R_\parallel e^{- I \sigma_3 p \cdot x} \text { and } \psi_2 = \rho^{1/2} I R_\parallel e^{I \sigma_3 p \cdot x},
  \end{equation} 

  which are CPT conjugates of one another. 

  $\psi_1$ and $\psi_2$ describe particles with the opposite spin, propagating in the same direction. In total, there are four solutions: two for Equation \ref{eq:even} which describe two particles of opposite helicity propagating in one direction, and two for Equation \ref{eq:odd} which describe two particles of opposite helicity propagating in the other direction. 

  The other two are given by $\psi_1' = \psi_1 \sigma_1$ and $\psi_2' = \psi_2 \sigma_1$, since $A=0$ and Equation \ref{eq:charge} leaves Equation \ref{eq:plane} invariant.

  \appendix

  \section{Matrix Formulation} \label{matrix}

  Following (8.70) in Reference \cite{gap}, Equations \ref{eq:even} and \ref{eq:odd} take the following form in matrix representation.

  \begin{equation}
    \hat \gamma^\mu (i \partial_\mu - e A_\mu) | \psi_+ \rangle = \omega_0 (I_4 + \hat \gamma_5) | \psi_+ \rangle
  \end{equation}

  and

  \begin{equation}
    \hat \gamma^\mu (i \partial_\mu - e A_\mu) | \psi_- \rangle = \omega_0 (I_4 - \hat \gamma_5) | \psi_- \rangle,
  \end{equation}

  where

  \begin{align}
    \gamma^0 = \begin{pmatrix} I_2 & 0 \\ 0 & -I_2 \end{pmatrix},\quad \gamma^k = \begin{pmatrix} 0 & \sigma^k \\ -\sigma^k & 0 \end{pmatrix},\quad \gamma_5 = \begin{pmatrix} 0 & I_2 \\ I_2 & 0 \end{pmatrix}.
  \end{align}


  \begin{thebibliography}{9}
    \bibitem{gap} 
      Doran and Lasenby.
      \textit{Geometric Algebra for Physicists}.
  \end{thebibliography}

\end{document}
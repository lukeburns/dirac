\documentclass{article}
\usepackage[margin=1.25in]{geometry}
\usepackage{amsmath}
\usepackage{amsfonts}

\title{An Extension of the Dirac Equation}
\author{Luke Burns}

\begin{document}
  \maketitle

  \section{Introduction}

  The free Dirac equation in STA is

  \begin{equation}
    \nabla \psi I \sigma_3 = \hat p \psi, \label{eq:dirac}
  \end{equation}

  where $\psi \in G_{1,3}^+$ such that\footnote{See GAP 8.86}

  \begin{equation}
    \hat p \psi = m \psi \gamma_0. \label{eq:mass}
  \end{equation} 

  A less restrictive constraint is just that

  \begin{equation}
    \hat p^2 = m^2
  \end{equation}

  be constant. With this lightened restriction, Equation \ref{eq:mass} can be replaced with

  \begin{equation}
    \hat p \psi = \psi p_0 \label{eq:momentum}
  \end{equation}

  where $p_0$ is a constant vector. This admits a new class of null solutions, where $\hat p^2 = 0$ but $\hat p \psi \not = 0$.\footnote{$\hat p$ is used in order to reserve $p$ for $p = R p_0 \widetilde R$.}

  \section{Symmetries}

  The extension of the full Dirac equation, including the electromagnetic gauge field

  \begin{equation}
    \nabla \psi I \sigma_3 - e A \psi = \psi p_0 \label{eq:full}
  \end{equation}

  is only invariant under the usual replacements $\psi \mapsto \psi e^{\alpha I \sigma_3}$ and $A \mapsto A - \nabla \alpha$ if 

  \begin{equation}
    p_0 \cdot I\sigma_3 = 0.
  \end{equation} 

  This is an essential feature of Dirac theory, so we'll restrict $p_0$ preserve it. We require that

  \begin{equation}
    p_0 = E \gamma_0 + |p| \gamma_3.
  \end{equation} 

  If $p_0^2 = m^2 \not= 0$, then 

  \begin{equation}
    p_0 = R m \gamma_0 \widetilde R
  \end{equation} 

  for $R = e^{\gamma_3 \gamma_0 \alpha/2}$ with $\alpha$ given by $\tanh(\alpha) = |p|/E$. Then $p_0 \mapsto \widetilde R p_0 R$ gives Equation \ref{eq:mass}.

  On the other hand, if $p_0^2 = 0$, then

  \begin{equation}
    p_0 = \omega (\gamma_0 + \gamma_3), \label{eq:massless}
  \end{equation}

  for $\omega = E = |p|$. Spinors satisfying Equation \ref{eq:momentum} with $p_0 = \omega(\gamma_0 + \gamma_3)$ are distinct from those satisfying Equation \ref{eq:mass}. This is a minimal extension that preserves the usual gauge field of Dirac theory, while admitting a new class of null solutions.\footnote{Do these relate to the null solutions of the Dirac equation?}

  Note that the conservation of the current density $J$ also depends on $p_0$.\footnote{GAP 8.93}

  \begin{align}
    \nabla \cdot J &= \langle \nabla \psi \gamma_0 \widetilde \psi \rangle + \langle \psi \gamma_0 \dot{\widetilde{\psi}} \dot \nabla \rangle\\
    &= 2 \langle I s (e A + p e^{-I \beta}) \rangle) \\
    &= 2 \langle I e^{-I \beta} s p \rangle \\
    &= 2 \sin(\beta) s \cdot p
  \end{align}

  where $s = \psi \gamma_3 \widetilde \psi$ and $p e^{-I\beta} = \psi p_0 \psi^{-1}$. See the \emph{Scratch Work} section for more details. This means that 

  \begin{equation}
    \nabla \cdot J = 0 \iff \gamma_3 \cdot p_0 = 0,
  \end{equation} 

  which holds for $p_0 = m \gamma_0$ and does not hold for $p_0 = \omega(\gamma_0 + \gamma_3)$.

  In the context of the normal Dirac equation, the conservation of $J$ is usually taken to mean that single fermions cannot be created or destroyed.\footnote{GAP p283} However, this interpretation may not make sense for null $p_0$. We'll try to make sense of these solutions next.

  \section{Scratch work}

  \subsection{$\nabla \cdot J$}

    \begin{align}
    \nabla \psi &= - (e A \psi + \psi p_0) I \sigma_3 \\
    \nabla \psi \gamma_0 \widetilde \psi &= - (e A \psi + \psi p_0) I \sigma_3 \gamma_0 \widetilde \psi \\
    &= - I (e A \psi + \psi p_0) \gamma_3 \widetilde \psi \\
    &= - I (e A + p e^{-I \beta}) s \\
    \langle \nabla \psi \gamma_0 \widetilde \psi \rangle &= - \langle I (e A + p e^{-I \beta}) s \rangle \\
    &= \langle I s e A \rangle +  \langle I s p e^{-I \beta}) \rangle
  \end{align}

  \begin{align}
    \dot{\widetilde{\psi}} \dot \nabla &= - (e \widetilde{A \psi I \sigma_3} + \widetilde{\psi p_0 I \sigma_3})\\
    &= e I \sigma_3 \widetilde{\psi} A + I \sigma_3 p_0 \widetilde{\psi} \\
    \psi \gamma_0 \dot{\widetilde{\psi}} \dot \nabla &= \psi \gamma_0 I \sigma_3 \widetilde{\psi} e A + \psi \gamma_0 I \sigma_3 p_0 \widetilde{\psi} \\
    &= I (\psi \gamma_3 \widetilde{\psi} e A + \psi \gamma_3 p_0 \widetilde{\psi})\\
    &= I (s e A + s \widetilde{\psi p_0 \psi^{-1}})\\
    &= Is (e A + p e^{-I \beta})\\
    \langle \psi \gamma_0 \dot{\widetilde{\psi}} \dot \nabla \rangle &= \langle I seA \rangle + \langle I s p e^{-I \beta} \rangle
  \end{align}

  where $p e^{-I \beta} = \psi p_0 \psi^{-1}$.

  \subsubsection{$\widetilde{\psi^{-1}} = \widetilde{\psi}^{-1}$}

  $\widetilde{\psi^{-1}} = \widetilde{\rho^{-1/2} e^{-I \beta/2} \widetilde R} = \rho^{-1/2} e^{-I \beta/2} R$

  $\widetilde{\psi}^{-1} = (\rho^{1/2} e^{I \beta/2} \widetilde R)^{-1/2} = \rho^{-1/2} e^{-I \beta/2} R$



\end{document}
\documentclass{article}
\usepackage[margin=1.5in]{geometry}
\usepackage{amsmath}

\title{Admitting Photons Into The Dirac Equation}
\author{Luke Burns}

\begin{document}
  \maketitle

  \section{Generalizing the Dirac Equation}

  The Dirac equation can be written 

  \begin{equation}
    \nabla \psi i = p \psi, \label{eq:dirac}
  \end{equation}

  where $\psi \in G_{1,3}^+$, $i$ is a constant unit bivector, and $p$ is a vector, such that\footnote{See GAP 8.86}

  \begin{equation}
    p \psi = m \psi \gamma_0. \label{eq:momentum}
  \end{equation} 

  A less restrictive constraint is just that

  \begin{equation}
    p^2 = m^2
  \end{equation}

  be constant. This admits a new class of null solutions, where $m = 0$ but $p \psi \not = 0$. 

  This paper aims to make sense of this new class of solutions.

\end{document}
